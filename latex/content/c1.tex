% LTeX: language=de-DE

\chapter*{Abstract}

\vspace{-2em}

\paragraph*{Englisch}
Passwords are considered to be the most common type of authentication. Despite their high establishment, they have a large number of vulnerabilities. For this reason, some passwordless alternatives for authentication have already been developed. One of these alternatives is the FIDO2 standard. It allows users to log in to online services using an authentication device. Frequently used authentication devices are security keys. These are an external piece of hardware that is only in the possession of the user. The FIDO2 standard is based on asymmetric cryptography, which allows authentication to be more secure than password-based authentication. This is especially due to the fact that FIDO2 credentials cannot be guessed and cannot be affected by data leaks or phishing emails. This thesis deals with the factors of security and usability of FIDO2 and with the integration of FIDO2 into Lufthansa Systems GmbH \& Co. KG.

\paragraph*{Deutsch}
Passwörter gelten seit langer Zeit als die gängigste Art der Authentifizierung. Trotz ihrer hohen Etablierung weisen sie eine Vielzahl an Schwachstellen auf. Aus diesem Grund wurden bereits einige passwortlose Alternativen zur Authentifizierung entwickelt. Eine dieser Alternativen ist der FIDO2 Standard. Er bietet Nutzern die Möglichkeit sich mithilfe eines Authentifizierungsgerätes bei Online-Diensten anzumelden. Häufig genutzte Authentifizierungsgeräte sind dabei Security Keys. Bei diesen handelt es sich um eine externe Hardware, welche sich nur im Besitz des Nutzers befindet. Der FIDO2 Standard basiert auf asymmetrischer Kryptografie, was eine Authentifizierung sicherer gestalten kann als eine passwortbasierte Authentifizierung. Dies liegt insbesondere daran, dass sich FIDO2 Zugangsdaten nicht erraten lassen und nicht von Datenlecks oder Phishing E-Mails betroffen seien können. Diese Arbeit beschäftigt sich mit den Faktoren der Sicherheit und Benutzerfreundlichkeit von FIDO2 und mit der Integration von FIDO2 in die Lufthansa Systems GmbH \& Co. KG.