% LTeX: language=de-DE

\chapter{Einführung}
Diese Arbeit beschäftigt sich mit passwortlosen Authentifizierungsverfahren. Im Folgenden werden zunächst die Problemstellung und das Ziel der Arbeit definiert. Anschließend wird der Aufbau der Arbeit beschrieben und auf verwandte Arbeiten eingegangen:

\section{Problemstellung \& Ziel der Arbeit} \label{target}
Die Problemstellung dieser Arbeit bezieht sich auf den passwortbasierten Ansatz der Authentifizierung, welcher im aktuellen Unternehmenskontext der \ac{LSY} implementiert ist. Aufgrund der hohen Etablierung handelt es sich bei der passwortlosen Authentifizierung, um den meist genutzten Verfahren Nutzer zu authentifizieren. Trotz der hohen Akzeptanz und Verbreitung weist das Verfahren einige schwerwiegende Sicherheitsbedenken auf. Dies kann zur Folge haben, dass Systeme leichter kompromittiert werden können und somit ein hoher Schaden für Unternehmen entstehen können.

Ziel dieser Arbeit ist es daher passwortlose Authentifizierungsverfahren als Alternative genauer zu betrachten. Verschiedene passwortlose Verfahren werden vorgestellt und ihre individuellen Vor- und Nachteile aufgezeigt. Dabei soll ein besonderes Augenmerk auf den Vergleich der Angriffsvektoren von passwortlosen und passwortbasierten Verfahren gelegt werden. Ein besonderer Fokus liegt auf der Analyse einer Nutzung von \ac{FIDO}2 in Kombination mit einem Security Key. Ziel der Analyse ist es eine Empfehlung abzuleiten, ob dieser alternative Ansatz für eine Nutzung innerhalb der \ac{LSY} geeignet ist und darzustellen, welche Anpassungen für eine Integration vorgenommen werden müssen. Betrachtet werden insbesondere die Aspekte der Sicherheit und der Benutzerfreundlichkeit. Ein weiterer Aspekt, welcher betrachtet wird ist die Frage, ob passwortlose Ansätze Passwörter ersetzen können oder lediglich ergänzen. 

\section{Aufbau der Arbeit}
Die Arbeit gibt zu Beginn zunächst eine Einführung in die verschiedenen Arten der Authentifizierung und stellt eine Verknüpfung zu den Schutzzielen der Informatik her. Anschließend wird die passwortbasierte Authentifizierung genauer betrachtet. Dabei werden die typischen Schwachstellen und Angriffsvektoren detailliert aufgezeigt und beschrieben. Auf Basis dieser Erkenntnisse werden mögliche passwortlose Alternativen vorgestellt. Diese werden kurz beschrieben und ihre Vor- und Nachteile betrachtet. Spezifischer wird auf die Nutzung von Security Keys (Yubikeys) eingegangen. Der Fokus liegt dabei auf der Recherche zum Thema der Benutzerfreundlichkeit. Anschließend wird das \ac{FIDO}2-Protokoll detailliert betrachtet. Die unterliegenden Protokolle WebAuthn und \ac{CTAP2} werden dargestellt und deren Funktionsweise erläutert. Hierbei werden ebenfalls der Aspekt der Sicherheit analysiert und die Unterschiede der passwortbasierten und passwortlosen Authentifizierung aufgezeigt. Als Grundlage dieser Ergebnisse dient eine wissenschaftliche Literaturrecherche.

Im weiteren Verlauf der Arbeit wird die aktuelle Lage der \ac{LSY} dargestellt. Der Fokus liegt dabei auf den aktuellen Prozessen der Authentifizierung. Um die Ergebnisse der Fachliteratur mit der Praxis zu vergleichen, wird eine Implementierung von \ac{FIDO}2 in Kombination mit einem Security Key innerhalb einer Abteilung der \ac{LSY} vorgenommen. Dabei wird betrachtet, ob und wie gut sich \ac{FIDO}2 in den Unternehmenskontext der \ac{LSY} integrieren lässt. Die vorgenommene Umsetzung wird aufgezeigt und mit den aktuellen Prozessen verglichen.

Um die Benutzerfreundlichkeit besser bewerten zu können wird auf Basis der zuvor erarbeiteten Ergebnisse ein interaktiver Fragebogen erstellt. Dieser wird innerhalb einer Abteilung der \ac{LSY} durchgeführt. Die Ergebnisse werden ausgewertet und mit den Ergebnissen der Fachliteratur verglichen. Ziel ist es dabei eine individuelle Empfehlung für die \ac{LSY} auszusprechen.

Abschließend spricht die Arbeit auf Basis der Literaturrecherche und der Auswertung des Fragebogens eine Empfehlung für die \ac{LSY} aus. Dabei wird die Frage beantwortet, ob passwortlose Authentifizierungsverfahren eine Alternative für die \ac{LSY} darstellen können.

\section{Referenzierte Arbeiten}

Bestandteil dieser Arbeit ist eine wissenschaftliche Literaturrecherche. Diese wird insbesondere genutzt, um die aktuelle Lage der Forschung zu passwortlosen Authentifizierungsverfahren zu ermitteln. Insbesondere werden Arbeiten betrachtet, welche sich mit der Nutzung von Passwörtern und deren Schwachstellen beschäftigen. Im Bereich der passwortlosen Authentifizierung werden insbesondere Arbeiten zu \ac{FIDO}2 und Security Keys detailliert betrachtet. Die Recherche wird mithilfe von Google Scholar und der Bibliothek der DHBW Mannheim durchgeführt. Dabei werden zu wichtigen Themen dieser Arbeit mehrere Arbeiten betrachtet, um die Aussagen der einzelnen Arbeiten zu verifizieren. Ebenfalls werden wissenschaftliche Arbeiten und Artikel aufgegriffen, welche sich in der Literatur als Grundlage etabliert haben und häufig zitiert werden. Da es sich bei der passwortlosen Authentifizierung um ein sehr aktuelles Thema handelt, ist eine Verifizierung der Aussagen teilweise nicht detailliert möglich. Im Folgenden werden die wichtigsten Arbeiten aufgeführt und kritisch reflektiert:

Die Arbeit \cite{boonkrong2012security} beschäftigt sich insbesondere mit den Grundlagen der Sicherheit von Passwörtern und deren Speicherung. Dabei werden die typischen Faktoren der Authentifizierung beschrieben. Diese bieten eine gute Grundlage für die Unterscheidung von passwortbasierten und passwortlosen Verfahren sowie von \ac{MFA} und \ac{SFA}. 

Für eine Einführung in die Schutzziele der Informatik wird die Arbeit \cite{samonas2014cia} als Grundlage genutzt. Diese ist eine häufig zitierte Arbeit, welche sich mit den Schutzzielen der Informatik und deren Entwicklung beschäftigt. Hier werden lediglich die Grundlagen der Schutzziele aus der Arbeit entnommen. Die Arbeit kann allerdings auch für einen tieferen Einblick genutzt werden.

Die Arbeit \cite{chanda2016password} beschäftigt sich insbesondere mit der Speicherung von Passwörtern und den benötigten Aufwand diese zu brechen. Für die durchgeführten Tests werden u.a. fünf und sechsstellige Passwörter verwendet. Diese sind in der Praxis nicht relevant, weil Passwörter eine höhere Länge aufweisen sollten. Dennoch wird der signifikante Zeitunterschied durch den simplen Unterschied der Passwortlänge deutlich. Auch die Unterschiede einer Veränderung des Zeichenraumes und der Passwortlänge wird betrachtet, welches eine wichtige Erkenntnis für die Sicherheit von Passwörtern darstellt.

Die Arbeit \cite{ives2004domino} betrachtet einen Dominoeffekt, welcher bei einer mehrfachen Verwendung eines Passwortes entstehen kann. Dies spielt eine große Rolle für die Sicherheit von Passwörtern und ist ein wichtiger Aspekt für die Betrachtung von passwortlosen Verfahren. Auch wenn das Paper bereits 2004 veröffentlicht wurde, ist dieser Dominoeffekt auch heute noch relevant und wird von vielen wissenschaftlichen Arbeiten zitiert.

Das Paper \cite{yildirim2019encouraging} beschäftigt sich mit Richtlinien für die Erstellung von Passwörtern. Für diese Arbeit wird lediglich die Erkenntnis des Papers genutzt, dass Richtlinien zu Erstellung von Passwörtern einen kontraproduktiven Effekt haben können. Für einen detailliertere Betrachtung von Maßnahmen, um Passwörter sicherer zu gestalten, empfiehlt sich ein Blick in das Paper. Es handelt sich ebenfalls um ein häufig zitiertes Paper.

Für die Einführung der passwortlosen Verfahren werden insbesondere die Arbeiten \cite{chowhan2019password} und \cite{parmar2022comprehensive} genutzt. Diese stellen einige passwortlose Verfahren vor und beschreiben deren Funktionsweise. Beide Arbeiten eignen sich für einen Einblick in weitere Alternativen zur passwortbasierten Authentifizierung. Für eine detaillierte Betrachtung der einzelnen passwortlosen Verfahren werden jedoch weitere Arbeiten benötigt.

Für eine Einführung in die Nutzung von Security Key und speziell YubiKeys wird das Datenblatt \cite{yuibkey2023fido2} genutzt. Dabei handelt es sich nicht um eine wissenschaftliche Arbeit, sondern ein Datenblatt des Herstellers Yubico. Dieses Datenblatt wird allerdings lediglich genutzt, um die Funktionen des YubiKeys aufzuführen. Es werden keine Aussagen über die Sicherheit oder die Benutzerfreundlichkeit aus dem Datenblatt übernommen.

Für die Betrachtung der Benutzerfreundlichkeit von Security Keys werden insbesondere die Ergebnisse der Arbeiten \cite{farke2020you}, \cite{lyastani2020fido2} und \cite{reynolds2018tale} genutzt. Es handelt sich dabei um Arbeiten, welche in verschiedenen Umfängen die Benutzerfreundlichkeit von Security Keys untersuchen, indem sie Nutzer befragen und Testphasen durchführen. Diese Ergebnisse werden in dieser Arbeit zusammengeführt, um eine die Benutzerfreundlichkeit von Security Keys zu bewerten. Ebenfalls dienen die Ergebnisse als Grundlage für die Erstellung des Fragebogens dieser Arbeit. So erfolgt eine wissenschaftliche Begründung für die Auswahl der Fragen und ermöglicht es ebenfalls die Ergebnisse der Arbeiten mit den Ergebnissen dieser Arbeit zu vergleichen. Bei der Arbeit \cite{reynolds2018tale} handelt es sich ebenfalls um eine etablierte Arbeit, welche auch in \cite{lyastani2020fido2} aufgeführt wird. 

Für die Betrachtung und Analyse der Funktionsweise von \ac{FIDO}2 werden ebenfalls die Arbeiten \cite{farke2020you} und \cite{lyastani2020fido2} genutzt. Der Großteil zur detaillierten Funktionsweise von WebAuthn und CTAP2/CTAP2.1 wird allerdings aus den Arbeiten \cite{barbosa2021provable} und \cite{bindel2022fido2} entnommen. Es handelt sich dabei um sehr aktuelle Arbeiten mit einem hohen wissenschaftlichen Anspruch. Die beiden Arbeiten sind die Grundsteine für die Analyse der Sicherheit von \ac{FIDO}2. Bei \cite{bindel2022fido2} handelt es sich zudem um eine Arbeit, welche Teilweise auf den Ergebnissen von \cite{barbosa2021provable} aufbaut.

Für eine wirtschaftliche Einschätzung der Kosten einer ausgenutzten Schwachstelle in Unternehmen wird \cite{databreach} verwendet. Dabei handelt es sich um eine Studie, welche die Kosten von Datenlecks in Unternehmen untersucht. Diese wird allerdings von IBM in Auftrag gegeben und kann daher nicht als komplett unabhängig betrachtet werden. Es handelt sich aber bei dem jährlich veröffentlichten Bericht um eine etablierte Studie, welche häufig in der Fachliteratur aufgegriffen wird. Da in dieser Arbeit allerdings nur ein kurzer Abschnitt auf die wirtschaftlichen Aspekte eingeht, wird die Studie als Richtwert verwendet. Für eine detailliertere Betrachtung der wirtschaftlichen Aspekte empfiehlt sich eine weitreichendere Recherche.

Da es sich bei dem Thema Passkeys um eine sehr aktuelle Technologie handelt, ist eine wissenschaftliche Recherche nur teilweise möglich. Da diese Arbeit lediglich einen Ausblick zum Thema Passkeys gibt, wird der Artikel \cite{usecasfido} zur Erklärung der Funktionsweise genutzt. Zum aktuellen Stand der Technologie werden die Dokumentationen von Apple \cite{passkeysapple} und Google \cite{passkeysgoogle} \cite{passkeysgoogledev} verwendet.