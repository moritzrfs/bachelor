\chapter{Einführung}
Diese Arbeit beschäftigt sich mit passwortlosen Authentifizierungsverfahren. Im Folgenden werden zunächst die Problemstellung und das Ziel der Arbeit erläutert. Anschließend wird der Aufbau der Arbeit beschrieben und auf verwandte Arbeiten eingegangen:

\section{Problemstellung \& Ziel der Arbeit}
Die Problemstellung dieser Arbeit bezieht sich auf den aktuellen, passwortlosen Ansatz der Authentifizierung im Unternehmenskontext der \ac{LSY}. Trotz ihrer hohen Etablierung und Verbreitung bieten passwortlose Authentifizierungsverfahren nicht nur Vorteile, sondern auch eine hohe Anzahl an Angriffsvektoren. 

Ziel dieser Arbeit ist es daher passwortlose Authentifizierungsverfahren als Alternative genauer zu betrachten. Verschiedene passwortlose Verfahren werden vorgestellt und ihre individuellen Vor- und Nachteile aufgezeigt. Dabei soll ein besonderes Augenmerk auf den Vergleich der Angriffsvektoren von passwortlosen und passwortbasierten Verfahren gelegt werden. Ein besonderer Fokus liegt auf der Analyse von FIDO2 in Kombination mit einem Security Key. Dabei wird analysiert, ob das Verafahren für die \ac{LSY} geeignet ist und welche Anpassungen vorgenommen werden müssen. Betrachtet werden insbesondere die Aspekte der Sicherheit und der Benutzerfreundlichkeit. Der Fokus liegt auf der Frage, ob passwortlose Verfahren eine Alternative darstellen, welche Passwörter gänzlich ersetzen. 

\section{Aufbau der Arbeit}
Die Arbeit gibt zunächst eine Einführung in die verschiedenen Arten der Authentifizierung und stellt eine Verknüpfung zu den Schutzzielen der Informatik her. Anschließend wird die Passwortbasierte Authentifizierung genauer betrachtet. Dabei werden die typischen Schwachstellen und Angriffsvektoren detailiert aufgezeigt und beschrieben. Auf Basis dieser Erkentnisse werden mögliche passwortlose Alternativen vorgstellt. Diese werden kurz beschrieben und ihre Vor- und Nachteile betrachtet. Spezifischer wird auf die Nutzung von Security Keys (Yubikeys) eingegangen. Ein Fokus liegt dabei auf der Recherche zum Thema der Benutzerfreundlichkeit. Anschließend wird das FIDO2-Protokoll detailiert betrachtet. Die unterliegenden Protkollo WebAuthn und CTAP2 werden dargestellt und deren Funktionsweise erläutert. Hierbei wird ebenfalls der Aspekt der Sicherheit analysiert. So sollen die Unterschiede der passwortbasierten und passwortlosen Authentifizierung verdeutlicht werden, insbesondere im Bezug auf die Sicherheit.

Nach der auf Fachliteratur basierenden Ergebnisse wird die aktuelle Lage der \ac{LSY} dargestellt. Der Fokus liegt dabei auf den aktuellen Prozessen der Authentifizierung. Um die Ergebnisse der Fachliteratur mit der Praxis zu vergleichen, wird eine Implementierung von FIDO2 in Kombination mit einem Security Key vorgenommen innerhalb einer Abteilung der \ac{LSY} vorgenommen. Dabei wird betrachtet, ob und wie gut sich FIDO2 in den Unternehmenskontext der \ac{LSY} integrieren lässt. Die vorgenommene Umsetzung wird aufgezeigt und mit den aktuellen Prozessen verglichen.

Um die Benutzerfreundlichkeit besser bewerten zu können wird auf Basis der erarbeiten Ergebnisse der Fachliteratur ein interaktiver Fragebogen erstellt. Dieser wird innerhalb einer Abteilung der \ac{LSY} durchgeführt. Die Ergebnisse werden ausgewertet und mit den Ergebnissen der Fachliteratur verglichen. Ziel ist es dabei eine individuelle Empfehlung für die \ac{LSY} auszusprechen.

Abschließend soll auf Basis der Literaturrecherche und der Auswertung des Fragebogens eine Empfehlung für die \ac{LSY} ausgesprochen werden. Dabei wird die Frage beantwortet, ob passwortlose Authentifizierungsverfahren eine aktuelle Alternative für die \ac{LSY} darstellen.
\section{Referenzierte Arbeiten}