\chapter{Einführung}
Diese Arbeit beschäftigt sich mit passwortlosen Authentifizierungsverfahren. Im Folgenden werden zunächst die Problemstellung und das Ziel der Arbeit erläutert. Anschließend wird der Aufbau der Arbeit beschrieben und auf verwandte Arbeiten eingegangen:

\section{Problemstellung \& Ziel der Arbeit}
Die Problemstellung dieser Arbeit bezieht sich auf den aktuellen, passwortlosen Ansatz der Authentifizierung im Unternehmenskontext der \ac{LSY}. Trotz ihrer hohen Etablierung und Verbreitung bieten passwortlose Authentifizierungsverfahren nicht nur Vorteile, sondern auch eine hohe Anzahl an Angriffsvektoren. 

Ziel dieser Arbeit ist es daher passwortlose Authentifizierungsverfahren genauer zu betrachten. Verschiedene passwortlose Verfahren werden vorgestellt und ihre Vor- und Nachteile aufgezeigt. Dabei soll ein besonderes Augenmerk auf den Vergleich der Angriffsvektoren von passwortlosen und passwortbasierten Verfahren gelegt werden. Einer der passwortlosen Verfahren wird begründet ausgewählt und detailiierter betrachtet. Dabei wird analysiert, ob das Verafahren für die \ac{LSY} geeignet ist und welche Anpassungen vorgenommen werden müssen. Betrachtet werden insbesondere die Aspekte der Sicherheit und der Benutzerfreundlichkeit. Der Fokus liegt auf der Frage, ob passwortlose Verfahren eine Alternative darstellen, welche Passwörter gänzlich ersetzen. 

\section{Aufbau der Arbeit}

\section{Referenzierte Arbeiten}