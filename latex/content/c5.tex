
\chapter{Fazit \& Empfehlung}
In Kapitel xy wurden insbesondere drei Ziele der Arbeit definiert. Diese werden im Folgenden bewertet. 
Die Ergebnisse der Arbeit machen deutlich, welche Schwachstellen und Angriffsvektoren aufzeigen. Aus diesem Grund existiert jedoch nicht bloß eine Alternative. Es gibt verschiedene passwortlose Ansätze, welche sich für verschiedene Anwendungsfälle eignen. Jeder Ansatz hat dabei seine Vor- und Nachteile. Deutlich wird jedoch, dass das FIDO2-Projekt zu den meist untestützten und am weitesten verbreiteten Ansätzen gehört. Dies liegt auch an der gebotenen Vielfalt, da FIDO2 nicht nur Security Keys, sondern beispielsweise auch Passkeys unterstützt.
Im Bezug auf Sicherheit ist das FIDO2-Protokoll eine erhebliche Verbesserung gegenüber der klassischen Passwortauthentifizierung. Da FIDO2 auf öffentliche/private Schlüssel basiert fallen die meisten Angriffsvektoren der klassischen Passwortauthentifizierung weg. 

Aus diesen Gründen empfiehlt sich die Integration von FIDO2 als Alternative ebenfalls für den Unternehmenskontext. Für die Nutzung von Security Keys hingegegen lässt sich keine eindeutige Empfehlung auf den gegebenen Kontext der \ac{LSY} aussprechen. Dies geht vor allem aus den Umsetzungsmöglichkeiten und der erarbeiteten Benutzerfreundlichkeit hervor. Aktuell ist eine FIDO2 Authentifizierung mit Hilfe eines Security Keys noch nicht ausreichend etabliert, um eine gesamte Umstellung vornehmen zu können. Nicht alle Dienste ermöglichen eine FIDO2 Authentifizierung. Zum aktuellen Zeitpunkt eignet sich die Nutzung von Security Keys lediglich für eine \ac{MFA}. Das Ergebnis dieser Arbeit ist allerdings, dass sich die Nutzung von Security Keys insbesondere für eine \ac{SFA} eignet.

Dies entspricht nicht den Richtlinien \ac{SFA}. Solange allerdings keine weitreichende Unterstützung von FIDO2 erfolgt, wird keine Änderung dieser Richtlinie empfohlen. Die Nutzung von Security Keys als zusätzlicher Faktor wird aus wirtschaftlichen Gründen nicht empfohlen. In diesem Fall ist die Nutzung einer Authenticator App besser geeignet.

Grundsätzlich ist eine Nutzung von FIDO2 als Alternative zur klassischen Passwortauthentifizierung zu empfehlen. Sobald eine ausschließliche Nutzung ermöglicht wird lassen siche erhebliche Vorteile für die Sicherheit erzielen. Lediglich die Nutzung von Security Keys ist nicht zweifelsfrei zu empfehlen. In dieser Arbeit werden mehrere Kritikpunkte an der Benutzerfreundlichkeit aufgezeigt. Fragwürdig ist, ob diese Kritikpunkte nach einer erweiterten Gewöhnungsphase noch bestehen. Eine Lösung könnte die Nutzung von Passkeys darstellen. Diese sind allerdings noch nicht weitreichend verbreitet.

Daraus folgt die Empfehlung für die \ac{LSY} nicht direkt auf FIDO2 in Kombination mit Security Keys zu setzen. Stattdessen sollte das Bewusstsein für passwortlose Alternativen erweitert werden. Die Offenheit für Alternativen sollte ebenfalls gefördert. Zusätzlich sollte die Etablierung von FIDO2 weiter beobachtet werden. Sobald eine ausschließliche Nutzung möglich ist, sollte eine Umstellung erfolgen.

\chapter{Ausblick Passkeys}